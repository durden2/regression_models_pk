\documentclass[12pt]{article}
\usepackage[english, polish]{babel}
\usepackage{polski}
\usepackage[utf8]{inputenc}
\usepackage{mathtools}
\usepackage{amsfonts}
\usepackage{amsmath}
\usepackage{amsthm}

\usepackage{csquotes}
% Ponieważ `csquotes` nie posiada polskiego stylu, można skorzystać z mocno zbliżonego stylu chorwackiego.
\DeclareQuoteAlias{croatian}{polish}

% Użyj czcionki kroju Courier.
\usepackage{courier}


\usepackage{listings}
\lstloadlanguages{TeX}

\lstset{
	literate={ą}{{\k{a}}}1
           {ć}{{\'c}}1
           {ę}{{\k{e}}}1
           {ó}{{\'o}}1
           {ń}{{\'n}}1
           {ł}{{\l{}}}1
           {ś}{{\'s}}1
           {ź}{{\'z}}1
           {ż}{{\.z}}1
           {Ą}{{\k{A}}}1
           {Ć}{{\'C}}1
           {Ę}{{\k{E}}}1
           {Ó}{{\'O}}1
           {Ń}{{\'N}}1
           {Ł}{{\L{}}}1
           {Ś}{{\'S}}1
           {Ź}{{\'Z}}1
           {Ż}{{\.Z}}1,
	basicstyle=\footnotesize\ttfamily,
}

% ------------------------

\AtBeginDocument{
	\renewcommand{\tablename}{Tabela}
	\renewcommand{\figurename}{Rys.}
}

% ------------------------
% --- < tabele > ---

\usepackage{array}
\usepackage{tabularx}
\usepackage{multirow}
\usepackage{booktabs}
\usepackage{makecell}
\usepackage[flushleft]{threeparttable}


\newcommand{\HRule}[1]{\rule{\linewidth}{#1}} 	% Horizontal rule

\makeatletter							% Title
\def\printtitle{%						
    {\centering \@title\par}}
\makeatother									

\makeatletter							% Author
\def\printauthor{%					
    {\centering \large \@author}}				
\makeatother							

% --------------------------------------------------------------------
% Metadata (Change this)
% --------------------------------------------------------------------
\title{	\normalsize \textsc{Modele Regresji} 	% Subtitle
		 	\\[2.0cm]								% 2cm spacing
			\HRule{0.5pt} \\						% Upper rule
			\LARGE \textbf{\uppercase{Problem regresji - Publiczny system wypozyczania rowerów}}	% Title
			\HRule{2pt} \\ [0.5cm]		% Lower rule + 0.5cm spacing
			\normalsize \today			% Todays date
		}

\author{
		Emilia Lubos\\
		Daria Pacewicz\\
		Michał Gandor\\		
}
\begin{document}
% ------------------------------------------------------------------------------
% Maketitle
% ------------------------------------------------------------------------------
\thispagestyle{empty}		% Remove page numbering on this page

\printtitle					% Print the title data as defined above
  	\vfill
\printauthor				% Print the author data as defined above
\section{Wstęp}

\hspace{7mm} Publiczny system wypożyczania rowerów jest nową alternatywą dla tradycyjnych wypożyczalni rowerów gdzie cały proces od członkostwa, wypożyczanie i zwrot rowerów jest zautomatyzowane. Dzięki takim systemom, użytkownik ma możliwość wypożyczenia roweru z jednego miejsca i zwrotu w innej stacji w bardzo prosty i wygodny sposób. Obecnie istnieje ponad 500 programów publicznego wypożyczania rowerów na całym świecie, na to składa się ponad 500 tysięcy rowerów.

W związku z tym generowane jest duża ilość danych, która można wykorzystać do ciekawych badań. W przeciwieństwie do innych środków transportu jak autobusy, tramwaje czy metro, długość podróży, miejsce startowe i docelowe jest zapisywane w systemie. Ta cecha zamienia system wypożyczania rowerów w wirtualny czujnik sieci komunikacji miejskiej, która może być użyta do zrozumienia mobilności w mieście.

\section{Opis zbioru danych}


Dane na temat wypożyczeń rowerów pochodzą z ''Capital Bikeshare system'' w mieście Washington D.C. w Stanach Zjednoczonych z lat 2011-2012. 

\begin{itemize}

\item instant - indeks rekordu,
\item dteday - data,
\item season - pora roku (1: wiosna, 2: lato, 3: jesień, 4: zima),
\item yr - rok (0: 2011, 1: 2012),
\item mnth - miesiąc (1 - 12),
\item hr - godzina (0 - 23),
\item holiday - czy dzień jest świętem czy nie,
\item weekday - dzień tygodnia,
\item workingday - jeśli dzień nie jest ani weekendem ani świętem = 1, w przeciwnym przypadku = 0, if day is neither weekend nor holiday is 1, otherwise is 0.
\item  weathersit - pogoda danego dnia: 
\begin{itemize}
	\item 1: Clear, Few clouds, Partly cloudy, Partly cloudy
	\item 2: Mist + Cloudy, Mist + Broken clouds, Mist + Few clouds, Mist
	\item 3: Light Snow, Light Rain + Thunderstorm + Scattered clouds, Light Rain + Scattered clouds
	\item 4: Heavy Rain + Ice Pallets + Thunderstorm + Mist, Snow + Fog
\end{itemize}
\item temp - Znormalizowana temperatura w stopniach Celsiusza, 
\item atemp - Znormalizowana temperatura odczuwalna w stopniach Celsiusza,
\item hum - Znormalizowana wilgotność,
\item windspeed - Znormalizowana prędkość wiatru,
\item casual - liczba zwykłych użytkowników,
\item registered - liczba zarejestrowanych użytkowników,
\item cnt - całkowita liczba wynajętych rowerów włączając zwykłych i zarejestrowanych użytkowników.

\end{itemize}

Całkowita liczba obserwacji wynosi 17 379.

\section{Cel projektu}

\hspace{7mm} Wypożyczanie rowerów jest bardzo powiązane z warunkami atmosferycznymi oraz z tym czy dzień jest wolny czy roboczy. Celem projektu jest stworzenie modelu regresji na bazie którego dokonana zostanie predykcja liczby wypożyczonych rowerów w zależności od zmiennych wymienionych powyżej oraz jednokierunkowej analizy wariancji która odpowie na pytania czy liczba wypożyczonych rowerów zależy od pory roku.
\section{Regresja wielokrotna}
W regresji zmiennymi objaśniającymi mogą być tylko zmienne ilościowe w związku z czym dla tego modelu zbiór został ogarniczony do 6 cech. Zmienne cnt jest sumą casual i registered dlatego też te dwie kolumny nie mogą być brane pod uwagę w rozpatrywanym modelu (występuje korelacja). Dane zostały poddane normalizacji. Po dokonaniu powyższych operacji współczynnik R2 wynosi 0.4609. Następnie dodana została nowa kolumna odpoiwadająca procentowej liczbie zarejestrowanych użytowników w stosunku do wszystkich wypożyczających. Współczynnik R2 wynosi 0.4602.
Kolejnym krokiem było usunięcie wartości odstających dla zmiennych temp, atemp, hum i windspeed. Ten zabieg zwiększył wartość współczynnika R2 do 0.4633.
Ostatnim etapem było usunięcie nieistotnych cech z użyciem metod eliminacji. Metoda bacward za zmienną nieistotną uznała zmienną 'temp', natomiast metoda forward nie wykluczyła żadnej z nich. Podjęta została decyzja o usunięciu wybranej nieistotnej kolumny, jednak nie poprawiło to ogólnego wyniku dopasowania danych do modelu regresji.
\section{Jednokierunkowa analiza wariancji}

\end{document}