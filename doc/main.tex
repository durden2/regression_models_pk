\documentclass[12pt]{article}
\usepackage[english, polish]{babel}
\usepackage{polski}
\usepackage[utf8]{inputenc}
\usepackage{mathtools}
\usepackage{amsfonts}
\usepackage{amsmath}
\usepackage{amsthm}

\usepackage{csquotes}
% Ponieważ `csquotes` nie posiada polskiego stylu, można skorzystać z mocno zbliżonego stylu chorwackiego.
\DeclareQuoteAlias{croatian}{polish}

% Użyj czcionki kroju Courier.
\usepackage{courier}


\usepackage{listings}
\lstloadlanguages{TeX}

\lstset{
	literate={ą}{{\k{a}}}1
           {ć}{{\'c}}1
           {ę}{{\k{e}}}1
           {ó}{{\'o}}1
           {ń}{{\'n}}1
           {ł}{{\l{}}}1
           {ś}{{\'s}}1
           {ź}{{\'z}}1
           {ż}{{\.z}}1
           {Ą}{{\k{A}}}1
           {Ć}{{\'C}}1
           {Ę}{{\k{E}}}1
           {Ó}{{\'O}}1
           {Ń}{{\'N}}1
           {Ł}{{\L{}}}1
           {Ś}{{\'S}}1
           {Ź}{{\'Z}}1
           {Ż}{{\.Z}}1,
	basicstyle=\footnotesize\ttfamily,
}

% ------------------------

\AtBeginDocument{
	\renewcommand{\tablename}{Tabela}
	\renewcommand{\figurename}{Rys.}
}

% ------------------------
% --- < tabele > ---

\usepackage{array}
\usepackage{tabularx}
\usepackage{multirow}
\usepackage{booktabs}
\usepackage{makecell}
\usepackage[flushleft]{threeparttable}


\newcommand{\HRule}[1]{\rule{\linewidth}{#1}} 	% Horizontal rule

\makeatletter							% Title
\def\printtitle{%						
    {\centering \@title\par}}
\makeatother									

\makeatletter							% Author
\def\printauthor{%					
    {\centering \large \@author}}				
\makeatother							

% --------------------------------------------------------------------
% Metadata (Change this)
% --------------------------------------------------------------------
\title{	\normalsize \textsc{Big data} 	% Subtitle
		 	\\[2.0cm]								% 2cm spacing
			\HRule{0.5pt} \\						% Upper rule
			\LARGE \textbf{\uppercase{Problem klasyfikacji - cukrzyca}}	% Title
			\HRule{2pt} \\ [0.5cm]		% Lower rule + 0.5cm spacing
			\normalsize \today			% Todays date
		}

\author{
		Emilia Lubos\\
		Daria Pacewicz\\
		Michał Gandor\\		
}
\begin{document}
% ------------------------------------------------------------------------------
% Maketitle
% ------------------------------------------------------------------------------
\thispagestyle{empty}		% Remove page numbering on this page

\printtitle					% Print the title data as defined above
  	\vfill
\printauthor				% Print the author data as defined above
\section{Opis zbioru danych}

Zbiór zawiera informacje czy u danego pacjenta występuje cukrzyca czy też nie. Pacjentami są kobiety w wieku 21 lat lub starszych pochodzących z Indii. Opis dokonany jest za pomocą zmiennych:

\begin{itemize}
\item Pregnancies - ilość ciąż,
\item Glucose - koncentracja glukozy wg 2-godzinnego testu,
\item BloodPressure - Ciśnienie krwi (mm Hg),
\item SkinThickness - gubość fałdu skóry na tricepsie (mm),
\item Insulin - 2-Hour serum insulin (mu U/ml)
\item BMI - index BMI (waga w kg/(wzrost w $m^2$)
\item DiabetesPedigreeFunction - funkcja rodowodu cukrzycy,
\item Age - wiek w latach,
\item Outcome - 0 = wynik negatywny (brak cukrzycy), 1 = wynik pozytywny (cukrzyca).
\end{itemize}

Rozkład klas:

\begin{itemize}
\item 0 - 500 próbek
\item 1 - 268 próbek
\end{itemize}

Całkowita liczba obserwacji wynosi 786.

\section{Cel projektu}

Celem projektu jest dokonanie klasyfikacji oraz zbadanie czy u danego pacjenta wystąpi cukrzyca czy nie. Zbadane zostanie także czy dane zawarte w~zbiorze są wystarczające do decyzji o prawdopodobieństwie wystąpienia choroby oraz czy wszystkie z nich wpływają znacząco na wystąpienie choroby.
\end{document}